\begin{abstract}
Radar technologies have recently emerged as a promising alternative to traditional odometry methods, which often rely on cameras, LiDAR, wheel encoders, inertial sensors, or GPS.  
Unlike these modalities, millimeter-wave (mmWave) radar offers robustness under adverse conditions such as fog, rain, or low light, while directly providing both range and Doppler velocity measurements.  
Previous work has demonstrated that radar can support odometry through global iterative closest point (ICP) alignment of point clouds.  
Building on this foundation, this work focuses on ego-motion estimation using a dual mmWave radar system complemented by an inertial sensor.  
The dual-radar configuration is designed to increase spatial coverage and reduce ambiguity in radar detections, addressing a key limitation of single-radar odometry approaches.  

Experimental evaluations were conducted in both enclosed environments (laboratory) and open environments (university parking lot).  
Results show that the proposed framework provides consistent and reliable ego-motion estimation, highlighting the potential of mmWave radar as a cost-efficient substitute or complement to conventional odometry sources in autonomous navigation.  
\end{abstract}

\begin{IEEEkeywords}
Radar Odometry, mmWave Radar, Dual-Radar Configuration, Clustering, ICP, Doppler Velocity, Submap Aggregation, Ego-Motion Estimation
\end{IEEEkeywords}
