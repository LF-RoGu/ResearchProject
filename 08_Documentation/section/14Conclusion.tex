\newpage
\section{Conclusion and Future Work}

This work presented a radar odometry framework using a dual IWR6843AOP mmWave radar setup for vehicle motion estimation.  
The two-sensor configuration showed a clear advantage in spatial awareness copared to single sensor arrangement, as it allowed a wider field of view and more consistent data for motion estimation.  
A detailed calibration process was carried out to configure the chirp frequencies of both radars.  
From the tests, it was confirmed that using identical chirp configurations on both sensors caused mutual interference, producing ghost detections and inconsistent Doppler values.  
This observation highlights the importance of proper frequency separation in multi-radar systems, since even small overlaps in frequency bands can lead to unstable measurements.

The proposed cluster-based odometry approach focused on selecting the most reliable tracked cluster over time to estimate ego-motion.  
This method worked effectively in outdoor environments such as the parking lot tests, where radar reflections were sparse and the overall point cloud was less stable.  
In contrast, indoor experiments showed that performing ICP on the full point cloud resulted in higher accuracy, as the denser reflections provided better point correspondences.  
These results suggest that both methods are useful depending on the environment: the cluster-based ICP is better suited for open outdoor areas, while the full point cloud ICP performs better in confined indoor spaces.

Overall, this project validated that mmWave radar sensors can be used successfully for odometry when supported by proper signal configuration and cluster tracking.  
The results demonstrate the feasibility of radar-based odometry as a foundation for more advanced applications such as SLAM.

\subsection{Future Work}

Future improvements will focus on refining the cluster tracking and data association process.  
A deeper analysis of the Doppler and angle-of-arrival information, for example by incorporating Doppler heatmaps, could improve the identification and consistency of static clusters.  
This would allow the system to better select stable targets and improve ego-motion estimation in more complex environments.

Another major step forward will be the integration of SLAM algorithms into the current framework.  
Since the cluster tracking system already distinguishes static objects and maintains their identities over time, these can be treated as landmarks for mapping.  
Combining radar odometry with SLAM would enable global localization, allowing the system to recognize and re-use previously observed landmarks even after they leave the radar’s field of view.  

Finally, future work will also include improving the visualization tool to support continuous frame processing instead of discrete jumps, ensuring smoother playback and more accurate analysis between frames.
