\subsection{"$x,y,z,\phi,SNR$" Filter}
The point cloud's points are passed through a first static filtering stage to remove points caused by noise, clutter or targets outside of the area of interest before being used for further processing.
This static filtering stage consists of four different filters, filtering out points by different attributes:
\begin{enumerate}
    \item Filtering by $SNR$: All points with a $SNR$ lower than \SI{12}{\deci\bel} are filtered out to remove points with a low signal and those that might be caused by noise or clutter.
    \item Filtering by $z$ coordinate: All points below a $z$ value of \SI{0}{\meter} and above \SI{2}{\meter} are filtered out to remove points caused by the ground or the ceiling.
    \item Filtering by $y$ coordinate: All points below a $y$ value of \SI{0.3}{\meter} are filtered out to remove points created by the driver's feet.
    \item Filtering by $\phi$: All points with an azimuth bigger than \SI{85}{\degree} are filtered out to remove points that are outside the area of interest.
\end{enumerate}
\textit{Note: The origin of the points' coordinate system is the sensor itself, so a coordinate of $(\SI{0}{\meter},\SI{0}{\meter},\SI{0}{\meter})$ is essentially at the sensor's mounting position and therefore approx. \SI{0.3}{\meter} above the ground.}
\par
As the static filtering stage only keeps points which are relevant in terms of there spatial position and $SNR$ for the following stages, it effectively decreases the computation time of each frame and prevents the following stages from processing invalid data.
The filtered point cloud is then passed to the next stages, the self-speed estimator and the dynamic filtering stage.

\begin{figure}[!htbp]
\centering
\begin{subfigure}{0.24\textwidth}
  \centering
  \includegraphics[width=\textwidth]{images/No_filter.png}
  \caption{Input}
\end{subfigure}
\begin{subfigure}{0.24\textwidth}
  \centering
  \includegraphics[width=\textwidth]{images/filter.png}
  \caption{Output}
\end{subfigure}
\caption{Example visualization of the input and output when filtering by the value of the $z$ coordinate.}
\label{fig:static_filter_z_example}
\end{figure}
\FloatBarrier\noindent

%% [20/03/2025] Leander: Rework done, commentend out bc. probably not needed anymore
%% ---------------------------------------
\begin{comment}
(effectively the radar sensor's mounting height as the coordinate's origin is at the sensor; approx. \SI{30}{\centi\meter} above ground) and \SI{2}{} 
Physical filtering further refines the data, this by removing points that can be considered as environmental noise or clutter, points outside the sensor’s effective monitoring area, or points whose radar cross-section (RCS) values fall below a predetermined threshold.
By mentioning the sensor effective monitoring area it is considered as area that the sensor will gather the information, process it and then give it as a result. But we might not be interested in all of this information.

\begin{figure}[!htbp]
\centering
\begin{subfigure}{0.24\textwidth}
  \centering
  \includegraphics[width=\textwidth]{images/filter.png}
\end{subfigure}
\begin{subfigure}{0.24\textwidth}
  \centering
  \includegraphics[width=\textwidth]{images/No_filter.png}
\end{subfigure}
\caption{The physical filters help us consider only information that may be caught by the sensor, but is of no use for us.}
\label{fig:remote_controller_components}
\end{figure}

For the project application and sensor location, it is considered that a Z-axis filter it is needed. As it will add more points that we will process if we don't discard them for the processing of the point-cloud.

This same concept may apply for other physical values of the point-cloud data, this being radial speed. If a point with a radial speed that is considered just ridiculous, then we add that point to the discarded data.
\end{comment}