\section{Introduction}
\label{sec:intoduction}

The advancement of autonomous systems and Advanced Driver Assistance Systems (ADAS) has increased the demand for robust and reliable perception technologies.
Among these, mmWave radar sensors have emerged as a promising solution due to their ability to operate effectively under various environmental conditions, such as low visibility, fog, rain, and darkness where traditional vision-based sensors, such as LiDAR or cameras, tend to have difficulties.
In those cases, where pure optical sensors do not maintain reliability due to the given operating conditions, mmWave radar sensors can still provide range, velocity, and angle information.
They therefore represent an essential component in modern perception frameworks, either as the only sensor that implements the solution or through a sensor-merge solution that compensates for the weaknesses of the various optical sensors.
\par

%% [14/03/2025] Luis: Rework of introduction
%% ---------------------------------------
%This paper presents a straightforward approach for implementing an object detection and emergency braking system for an electric go-kart by using Texas Instrument's IWR6843AOPEVM mmWave radar sensor and a modular pipeline consisting of different building blocks for real-time data processing.
%These building blocks cover every needed step from the processing of the radar sensor's raw point cloud to the initiation of an emergency braking event. This is done by utilizing techniques, such as various filtering techniques and a radar-only ego-velocity (the vehicle's own velocity) estimation for the separation of static and moving objects together with clustering algorithms mapping to occupancy grids in cartesian and polar coordinates for object detection and tracking.

This paper presents a straightforward approach on how to implement an object detection and emergency braking system for an electric go-kart by utilizing Texas Instrument's IWR6843AOP mmWave radar sensor.
The implementation involves a modular processing pipeline composed of several building blocks which are designed for real-time data processing.
These components encompass the entire process: from the data-acquisition and initial processing of raw radar point cloud data to the final triggering of emergency braking events.

The approach integrates multiple filtering techniques, radar-based self-speed estimation (estimation of the vehicle's own velocity) together with clustering algorithms. These techniques are applied to effectively differentiate static from moving objects and enable the precise detection and tracking of objects.

In particular, static and dynamic filtering stages are combined with a two-stage clustering block to ensure a reliable detection of static objects.
The static filtering techniques use the point cloud's  $SNR$ information and spatial coordinates to refine the data in the pipeline's early stages by discarding points that are identified as clutter or those that are simply outside the monitored area.
Doppler velocity measurements of the points compared to the vehicle's self-speed enable the differentiation between static and dynamic objects, improving overall tracking capabilities and enabling accurate motion predictions.


%In this project, the capabilities of Texas Instrument's IWR6843AOP mmWave radar sensor are used to implement an object detection and emergency braking system for an electric go-kart by %utilizing object detection, tracking, and environmental mapping in a highly reflective zone \todo{What is meant by highly reflective zone?}.
%The presented approach integrates various filtering techniques and a radar-only self-speed (the vehicle's own velocity) estimation for the separation of static and moving objects together with clustering algorithms and mapping to occupancy grids in cartesian and polar coordinates for object detection and tracking.


%% Commented for reworking (03/08/2025)
%% LH: I like the way of how the introduction starts but the text was somewhat circular (ADAS --> mmWave --> our project --> mmWave) and we need to avoid first person pronouns "we"

%Our approach uses dynamic filtering techniques cluster-based object identification, dynamic filtering techniques, and occupancy grid %mapping to improve detection accuracy and differentiate real objects from noise.
%\par
%To enhance the robustness of our object detection, we apply physical filtering techniques at the beginning of our pipeline to refine the radar point-cloud data, so that we do not use points that we know will be considered as noise beforehand.
%Additionally, Doppler velocity estimation is used to analyze both object motion and ego-velocity (the vehicle's own movement), contributing to better tracking and motion prediction, so we can classify object into 2 main categories, moving objects and static objects.
%\par
%Now, with that in mind, as a final introduction to mWave radar technology, this technology has found widespread applications in various domains, including autonomous vehicles, robotics, and industrial safety systems. Its ability to operate reliably under challenging environmental conditions makes it an essential component for real-time perception and object detection. Using advanced clustering, filtering techniques, and occupancy grid mapping, mmWave sensors can provide a more comprehensive understanding of dynamic environments, improving safety and navigation efficiency. As radar technology continues to evolve, the integration of machine learning and sensor fusion techniques further enhances its capabilities, enabling smarter, more adaptive perception systems for next-generation autonomous applications.