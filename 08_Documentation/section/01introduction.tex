\section{Introduction}
\label{sec:intoduction}

Accurate and reliable ego-motion estimation is a fundamental requirement for mobile robotic systems and autonomous vehicles solutions.
Most of the current odometry implementations rely on vision- or LiDAR-based sensors, which provide dense environmental information.
However, these modalities often suffer from high cost and performance degradation under adverse conditions such as low illumination, fog, rain, or snow.
In such scenarios, the accuracy and robustness of odometry are significantly reduced, raising the demand for complementary sensing solutions.

Odometry provides the basis for localization, mapping, and navigation, serving as a core component in modern perception solutions.  
In most scenarios, odometry has been implemented using vision- or LiDAR-based sensors, which offer dense information about the environment.  
However, these modalities often suffer from performance degradation from high memory consumption and being under adverse conditions such as low illumination, fog, rain, or snow.  
In such scenarios, the accuracy and robustness of odometry are significantly limited and reduced, raising the demand for complementary sensing solutions.  

Millimeter-wave (mmWave) radar has emerged as a promising candidate to address these challenges.  
Radar sensors are compact, cost-efficient, and resilient to weather and lighting variations, making them attractive for robotic and automotive applications.  
Unlike vision or LiDAR, radar directly measures range and Doppler velocity, providing unique information for motion estimation.  
Nevertheless, radar data presents notable challenges: the resulting point clouds are sparse, noisy, and subject to reflections from dynamic objects, complicating the task of reliable odometry.  
This problem that creates an unreliable and unstable odometry calculation, which is a critical issue for autonomous systems that require precise and consistent motion calculation, we can say that what is obtained is an estimation.
With the goal of obtaining an accurate estimation.

This work explores the use of a single front-mounted mmWave radar sensor to estimate vehicle ego-motion without relying on additional sensor modalities.  
To achieve this, an implementation for processing that integrates clustering methods to structure radar detections, Doppler velocity augmentation to exploit velocity measurements, and point-to-point iterative closest point (ICP) optimization for scan-to-scan registration.  
A submap-based approach is further applied to increase stability by aggregating multiple frames, thereby improving alignment despite noise and sparsity.  
Importantly, the point-to-point ICP method eliminates the need for an initial pose guess, which is difficult to obtain in radar data, making it particularly suitable for this application.  

The contributions of this work can be summarized as follows:  
\begin{enumerate}
    \item A radar-only odometry using a single mmWave sensor, minimizing hardware cost and complexity.  
    \item Integration of Doppler velocity information into the registration process to enhance accuracy.  
    \item Use of submap aggregation to mitigate sparsity and improve robustness in dynamic environments.  
    \item Object tracking via clusters to identify and filter dynamic objects from the ego-motion estimation.
    \item Experimental evaluation of the proposed approach under realistic driving conditions.  
\end{enumerate}