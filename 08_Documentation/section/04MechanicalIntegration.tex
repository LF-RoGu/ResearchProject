\subsubsection{Mechanical Integration}
All sensors were mounted on a Ninebot Go-Kart test vehicle.  
Custom 3D-printed mounts were designed to ensure rigid placement and repeatability of radar orientation.  
The radars were positioned at the front of the vehicle with partially overlapping fields of view to improve coverage and reduce blind spots.  
The IMU was installed near the center of the kart to minimize rotational offsets.  
The webcam was mounted facing forward to log experiments for later analysis.  

\subsubsection{Radar Chirp Configuration}
Radar performance is strongly determined by chirp parameters.  
For odometry, the configuration aimed to balance:  
\begin{itemize}
    \item \textbf{Range resolution} - to separate nearby objects.  
    \item \textbf{Update rate} - to maintain real-time odometry.  
    \item \textbf{Field of view} - to capture sufficient static structures.  
\end{itemize}

This setup ensured compatibility with the subsequent pipeline stages, particularly Doppler filtering, clustering, and ICP alignment.  
While the exact numerical values are described later in the experimental section, the conceptual rationale here highlights how chirp design affects odometry robustness.  
