\subsection{Iterative Closest Point (ICP) for Ego-Motion Estimation}
\label{subsec:icp}

The Iterative Closest Point (ICP) algorithm estimates the rigid-body transformation 
between two consecutive radar frames. Given two point sets
\[
P = \{p_i \in \mathbb{R}^2 \}_{i=1}^{N}, \quad
Q = \{q_j \in \mathbb{R}^2 \}_{j=1}^{M},
\]
the goal is to find a rotation $R \in SO(2)$ and translation $t \in \mathbb{R}^2$ that minimize the alignment error
\[
\min_{R,t} \sum_{i=1}^N \lVert R p_i + t - q_{\pi(i)} \rVert^2,
\]
where $\pi(i)$ denotes the index of the nearest neighbor of $p_i$ in $Q$.

\subsubsection*{Nearest Neighbor Matching}
To associate points between frames, a KD-tree was constructed over the set $Q$.  
For each point $p_i \in P$, its closest match $q_{\pi(i)}$ is obtained as:
\[
\pi(i) = \arg\min_j \lVert p_i - q_j \rVert_2.
\]

\subsubsection*{Per-Point Motion Estimates}
For each matched pair $(p_i, q_{\pi(i)})$, the incremental translation and orientation change are computed as
\[
t_i = p_i - q_{\pi(i)}, \qquad
\theta_i = \arctan2\!\left( (p_i - q_{\pi(i)})_y, (p_i - q_{\pi(i)})_x \right).
\]

\subsubsection*{Global Averaging}
The global motion is obtained by averaging over all matched pairs:
\[
\bar{t} = \frac{1}{N} \sum_{i=1}^N t_i, \qquad
\bar{\theta} = \frac{1}{N} \sum_{i=1}^N \theta_i.
\]

\subsubsection*{Homogeneous Transformation Matrices}
The estimated transformation from ICP is expressed in homogeneous coordinates as:
\[
T_{\text{ICP}} =
\begin{bmatrix}
\cos\bar{\theta} & -\sin\bar{\theta} & \bar{t}_x \\
\sin\bar{\theta} &  \cos\bar{\theta} & \bar{t}_y \\
0 & 0 & 1
\end{bmatrix}.
\]

For ego-motion estimation (inverse motion), the corresponding matrix is:
\[
T_{\text{ego}} =
\begin{bmatrix}
\cos\bar{\theta} & \sin\bar{\theta} & -(\bar{t}_x \cos\bar{\theta} + \bar{t}_y \sin\bar{\theta}) \\
-\sin\bar{\theta} & \cos\bar{\theta} & \;\;\bar{t}_x \sin\bar{\theta} - \bar{t}_y \cos\bar{\theta} \\
0 & 0 & 1
\end{bmatrix}.
\]

\subsubsection*{Cluster-wise ICP}
In addition to the global ICP (using all points), a cluster-wise ICP is applied.  
Each cluster $C_k$ maintains counters for \emph{hits} and \emph{misses}, ensuring stable targets are prioritized.  
For each cluster, the transformation $(t_k, \theta_k)$ is estimated using the above procedure.  
The global motion estimate is then computed as a weighted average:
\[
\bar{t} = \frac{\sum_k w_k t_k}{\sum_k w_k}, \qquad
\bar{\theta} = \frac{\sum_k w_k \theta_k}{\sum_k w_k},
\]
with weights
\[
w_k = \max\!\left(\min\!\left(\text{hits}_k^{\alpha}, \; \text{max\_hits}\right), 1\right),
\]
where $\alpha \in [1,2]$ controls the influence of stable clusters.

\subsubsection*{Python-style Conceptual Flow}
\begin{enumerate}
    \item Input two frames $P$ and $Q$.
    \item Use a KD-tree to find nearest neighbors between $P$ and $Q$.
    \item For each matched pair $(p_i,q_{\pi(i)})$, compute translation $(t_x,t_y)$ and rotation $\theta$.
    \item Average all translations and rotations to obtain $(\bar{t},\bar{\theta})$.
    \item Build the homogeneous transformation $T_{\text{ICP}}$.
    \item Derive the inverse ego-motion matrix $T_{\text{ego}}$.
    \item (\textbf{Cluster ICP}) Repeat steps 2–6 on each cluster and fuse results using hit-based weighting.
\end{enumerate}
