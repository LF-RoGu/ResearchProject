\subsubsection{Iterative Closest Point (ICP)}
\label{subsec:icp}

The Iterative Closest Point (ICP) algorithm is a classical method for estimating rigid-body transformations between two point sets. 
In the context of radar odometry, ICP provides a way to align consecutive frames of radar point clouds in order to estimate ego-motion. 
The main goal is to determine the relative rotation $R \in SO(2)$ and translation $t \in \mathbb{R}^2$ that minimize the alignment error between a source point cloud $P = \{p_i\}_{i=1}^N$ and a target point cloud $Q = \{q_j\}_{j=1}^M$.

\paragraph{Mathematical Formulation}
The ICP algorithm can be separated into two main steps: correspondence search and transformation estimation.

\textbf{Correspondence Search.}
For each point $p_i \in P$, the closest point in $Q$ is selected according to the Euclidean distance:
\begin{equation}
    q^*_i = \arg \min_{q_j \in Q} \lVert p_i - q_j \rVert_2.
\end{equation}
This step establishes a set of tentative correspondences $\{(p_i, q^*_i)\}$ between the two frames. 
Efficient implementations use spatial data structures such as KD-trees to accelerate nearest-neighbor searches.

\textbf{Transformation Estimation.}
Once correspondences are found, the rigid-body transformation $(R,t)$ is estimated by minimizing the least-squares error:
\begin{equation}
    E(R,t) = \sum_{i=1}^{N} \lVert p_i - (R q^*_i + t) \rVert^2.
\end{equation}
The optimal solution can be obtained using singular value decomposition (SVD) of the cross-covariance matrix between the two point sets. 
In 2D, the rotation can be parameterized by an angle $\theta$, such that:
\begin{equation}
    R(\theta) = 
    \begin{bmatrix}
        \cos\theta & -\sin\theta \\
        \sin\theta & \cos\theta
    \end{bmatrix}.
\end{equation}

The transformation is then applied to $Q$, and the process is repeated iteratively until convergence, typically when changes fall below a threshold or a maximum number of iterations is reached.

\paragraph{Global ICP}
In the global variant, all available points of the radar point cloud are used to compute correspondences and estimate the motion. 
This maximizes the use of sensor information but can be sensitive to noise and dynamic objects, since outliers directly affect the transformation estimate.

\paragraph{Cluster-wise ICP}
To increase robustness, a cluster-wise approach was also investigated. 
Instead of using the entire point cloud, correspondences are restricted to clusters identified in the previous stage (see Section~\ref{subsec:cluster_tracking}). 
Among these clusters, priority is given to those with the highest \emph{hit count} and lowest \emph{miss count}, since they are more likely to represent stable, stationary objects. 
For each cluster $C_k$, an independent ICP alignment is performed:
\begin{equation}
    E_k(R,t) = \sum_{p_i \in P_k} \lVert p_i - (R q^*_i + t) \rVert^2,
\end{equation}
where $P_k$ and $Q_k$ denote corresponding cluster subsets across two consecutive frames.
The resulting transformations are then aggregated, typically via weighted averaging based on cluster stability, to obtain the final ego-motion estimate.

\paragraph{Practical Considerations}
Several practical aspects influence ICP performance:
\begin{itemize}
    \item \textbf{Initialization:} ICP assumes small inter-frame motion. Poor initialization may cause convergence to local minima.
    \item \textbf{Nearest-neighbor search:} KD-trees enable efficient correspondence search in $\mathcal{O}(N \log N)$ time.
    \item \textbf{Outlier rejection:} Correspondences with distances exceeding a threshold $d_{\max}$ are discarded to reduce the influence of spurious points.
    \item \textbf{Cluster selection:} Restricting ICP to stable clusters improves robustness against moving objects and noise.
\end{itemize}
