\subsection{Iterative Closest Point (ICP)}
\label{subsec:icp}

The Iterative Closest Point (ICP) algorithm is a classical method for estimating rigid-body transformations between two point sets. 
In the context of radar odometry, ICP provides a way to align consecutive frames of radar point clouds in order to estimate ego-motion. 
The main goal is to determine the relative rotation $R \in SO(2)$ and translation $t \in \mathbb{R}^2$ that minimize the alignment error between a source point cloud $P = \{p_i\}_{i=1}^N$ and a target point cloud $Q = \{q_j\}_{j=1}^M$.

\paragraph{Mathematical Formulation}
The ICP algorithm can be separated into two main steps: correspondence search and transformation estimation.

\textbf{Correspondence Search.}
For each point $p_i \in P$, the closest point in $Q$ is selected according to the Euclidean distance:
\begin{equation}
    q^*_i = \arg \min_{q_j \in Q} \lVert p_i - q_j \rVert_2.
\end{equation}
This step establishes a set of tentative correspondences $\{(p_i, q^*_i)\}$ between the two frames. 
Efficient implementations use spatial data structures such as KD-trees to accelerate nearest-neighbor searches.

\textbf{Transformation Estimation.}
Once correspondences are found, the rigid-body transformation $(R,t)$ between two point sets is estimated by minimizing the least-squares error:
\begin{equation}
    E(R,t) = \sum_{i=1}^{N} \lVert p_i - (R q^*_i + t) \rVert^2.
\end{equation}

Here, $R$ denotes the rotation matrix, $t$ is the translation vector, and together $(R,t)$ form the rigid-body transformation that aligns the source points $q^*_i$ with the target points $p_i$.  
This transformation preserves distances and angles, meaning objects are only rotated and shifted, but not deformed.  

The optimal solution is obtained using singular value decomposition (SVD) of the cross-covariance matrix between the two point sets.  

In 2D, the rotation $R$ can be parameterized by a single angle $\theta$ as:
\begin{equation}
    R(\theta) = 
    \begin{bmatrix}
        \cos\theta & -\sin\theta \\
        \sin\theta & \cos\theta
    \end{bmatrix}.
\end{equation}


\paragraph{Global ICP}
In the global variant, all available points of the radar point cloud are used to compute correspondences and estimate the motion. 
This maximizes the use of sensor information but can be sensitive to noise and dynamic objects, since outliers directly affect the transformation estimate.

\paragraph{Cluster-wise ICP}
To increase robustness, a cluster-wise approach was also investigated.  
Instead of using the entire point cloud, correspondences are restricted to clusters identified in the previous stage (see Section~\ref{subsec:cluster_tracking}).  

Within these clusters, the algorithm searches for the most reliable candidate, defined as the one with the \textbf{highest hit count} and the \textbf{lowest miss count}.  
This reliability criterion favors clusters that have been consistently observed across frames and are less likely to represent transient or dynamic objects.  

Once the most reliable cluster $C_k$ is selected, an independent ICP alignment is performed:
\begin{equation}
    E_k(R,t) = \sum_{p_i \in P_k} \lVert p_i - (R q^*_i + t) \rVert^2,
\end{equation}
where $P_k$ and $Q_k$ denote the corresponding subsets of cluster points across consecutive frames.  

The estimated transformation $(R,t)$ from this cluster is then used directly as the ego-motion update, under the assumption that stable clusters best represent static landmarks in the environment.  
This strategy reduces the influence of noisy or dynamic clusters and improves odometry consistency, especially in outdoor scenarios.

\paragraph{Practical Considerations}
Several practical aspects influence ICP performance:
\begin{itemize}
    \item \textbf{Initialization:} ICP assumes small inter-frame motion. Poor initialization may cause convergence to local minima.
    \item \textbf{Nearest-neighbor search:} KD-trees enable efficient correspondence search in $\mathcal{O}(N \log N)$ time.
    \item \textbf{Outlier rejection:} Correspondences with distances exceeding a threshold $d_{\max}$ are discarded to reduce the influence of spurious points.
    \item \textbf{Cluster selection:} Restricting ICP to stable clusters improves robustness against moving objects and noise.
\end{itemize}
